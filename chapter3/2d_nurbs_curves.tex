\section{2D NURBS curves}
% \subsection{Shape functions}
% \label{subsection:shape_functions}
%     \begin{equation}
%         123
%     \label{eq:rational_basis_function}
%     \end{equation}

% \subsection{Curve representation}

% \subsection{Basic geometric algorithm}

\subsection{Numerical integrations}
\label{subsection:numerical_integration}
\paragraph{}
When computing the integrations in SBFEM % equation
    , numerical integrations tends to be overwhelmingly preferred over mathematical deduction. 
The reason behind lies in the flexibility of the numerical and that deduction of exact integrations scheme 
    to any given shape functions are not feasible. 
Due to the fact that the polynomials are adopted as the shape function, the numerical integrations methods 
    such as Legendre Quadrature or Gauss Quadrature provides possibility for an exact integration. 
An integration quadrature is normally defined as followed:
    \begin{equation}
        \int_{-1}^{1}
        f(x)dx 
        = \sum_{i=1}^n
        a_i f(x_i)
    \label{eq:numerical_integration}
    \end{equation}

\paragraph{}
Any given targeted polynomial function defined on $[-1,1]$ can be explicitly expressed as series.
A set of integration points $\left\{ x_1, x_2, \dots, x_n \right\} \in \left[-1,1\right]$ and the corresponding weight $\left\{ a_1, a_2, 
    \dots, a_n \right\} \in \mathbb{R}$ determined from the integration quadrature can be adopted to perform an exact integration
    on the given function.
\paragraph{}
Although shape functions used in NURBS are not polynomials, they can be separated into several spans where the function is a 
    rational polynomial. 
Based on this property, we are able to apply the numerical integration quadrature on each of these spans and achieve a reasonably 
    accurate result.
In other words, the NURBS curve with a knot vector of 
$[ 
    \underbrace{-1,-1,\dots,-1}_{p+1}, 
    u_0,\dots,u_n, 
    \underbrace{1,1,\dots,1 }_{p+1}
]$
can be integrated as
\begin{equation}
    \int_{-1}^{1} R(u) du = \int_{-1}^{u_0} R(u)du + 
                            \int_{u_0}^{u_1} R(u)du + \dots +
                            \int_{u_n}^1 R(u)du
\label{eq:numerical_integration_piecewise}
\end{equation}

\paragraph{}
Since the rational polynomials instead of the usual polynomials are utilized as the shape functions in NURBS, the difference between
    output from eq.~\ref{eq:numerical_integration_piecewise} and the analytical solution will be so large that can not be regarded as
    machine error.
Based on eq.~\ref{eq:rational_basis_function} we can conclude that the basis functions constructed by rational polynomials become
    non-rational if and only if the weight vector is identical i.e. $\left\{ w \right\} = \left[ 1,1,\dots,1 \right]$ after normalization.
That indicates the error of numerical integration will be decreased when the weight vector of the NURBS curves becomes more uniform as
    the basis functions are more close to non-rational polynomials.
In order to achieve this target, either or both of the knot insertion or the order elevation can be used.
\pagebreak
% ===================================================================================================================================== %

\subsection{Surface traction}
\label{subsection:surface_traction}
\paragraph{}
In structural analysis, it is common to have boundary condition such as displacement constraints and applied load.
Due to the property of the NURBS that the control points are not necessarily on the curve, surface traction can not be
    applied by same method used in conventional numerical method like FEM or SBFEM.
A surface traction $\Phi$ can be regarded as Neumann boundary condition which can be expressed as
    \begin{equation}
        {F}=-\int_{\Gamma}
        [N]
        \Phi_n
        d\Gamma
    \label{eq:neumann_bc}
    \end{equation}
where $[N]$ describe the shape functions and $\Phi_n$ is the surface traction on the nodes.

\paragraph{}
As mentioned in \ref{subsection:numerical_integration}, numerical integration would be much more preferred over mathematical deduction
when the target function is an input. In the flavour of numerical integration, eq.~\ref{eq:neumann_bc} can be expressed as followed.
    \begin{equation}
        {F}=-\sum_{i=1}^n
        a_i
        [N(\xi_i)]
        \Phi_n
    \label{eq:neumann_bc_numerical}
    \end{equation}
Where $\xi_i$ is the integration points and $a$ is the weights,
$n$ is the number of integration points and different quadrature rule need different number to achieve a optimal accuracy.

\paragraph{}
It can be found that the term $[N(\xi_i)] \Phi_n$ is corresponding to $f(x)$ in eq.~\ref{eq:numerical_integration}.
In conventional FEM or SBFEM, $\Phi_n$ can be determined as the real values on the nodes because geometrically speaking,
    its shape function is interpolated from the given set of points.
In other words, all nodes that determine the shape function in traditional FEM or SBFEM must be on the interpolating function.
However, this is not the case in NURBS curves where it is the control points that play the same role as the nodes in existing
    shape function.
    % figure required
In NURBS curves, apart from the first and the last points, the control points are not necessarily on the curves.
This prevent us from adopting the physical value on the nodes as $\Phi_n$ in eq.~\ref{eq:neumann_bc_numerical}.
Instead, a set of ``control stress'' $\Phi_c$, the control points of another NURBS curve that represent the surface traction
    geometrically, need to be determined as \footnote{$\argmin_x f(x) = \left\{
        x | x \in S \wedge \forall y \in S : f(y) \geq f(x)
    \right\}$}
    \begin{equation}
        \Phi_c = \argmin_{\Phi_c}
            \frac{1}{2}
            \int_{-1}^1
            \|
                \Phi(\xi)-
                    \left[ N(\xi) \right]
                    \Phi_c
            \|^2
            d\xi            
    \label{eq:surface_traction_fitting}
    \end{equation}

\paragraph{}
It means that ``control stress'' $\Phi_c$ describe a minimum mean squared error between surface traction NURBS curve and the real
    traction $\Phi$.
One of the simplest mathematical method to determine $\Phi_c$ will be least square method.
Given the fact that the shape functions of this NURBS curve will be the same as that describe the geometry, $\left[ N(\xi) \right]$
    can be considered as known.
By selecting $n$ sample points over the domain of the $\Phi$, eq.~\ref{eq:surface_traction_fitting} can be rewrite as
    \begin{equation}
        \Phi_c = \argmin_{\Phi_c}
            \frac{1}{n}
            \sum_{i=1}^n
            \|
                \Phi(\xi_i)-
                    \left[ N(\xi_i) \right]
                    \Phi_c
            \|^2
    \label{eq:surface_traction_fitting_discrete}
    \end{equation}
Then ``control stress'' $\Phi_c$ can be solved by least square as
    \begin{equation}
        \Phi_c= \left(
            \left[ N(\xi) \right] ^T
            \left[ N(\xi) \right]
        \right)^{-1}
        \left[ N(\xi) \right]^T
        \Phi(\xi)
    \end{equation}
and eq.~\ref{eq:neumann_bc_numerical} in the case where NURBS is in use can be rewrite as
    \begin{equation}
        {F}=-\sum_{i=1}^n
        a_i
        [N(\xi_i)]
        \Phi_c
    \label{eq:neumann_bc_numerical_NURBS}
    \end{equation}
\pagebreak
% ===================================================================================================================================== %