\subsection{Numerical integrations}
\label{iso_subsection:numerical_integration}
\paragraph{}
When computing the integrations in SBFEM % equation
    , numerical integrations tends to be overwhelmingly preferred over mathematical deduction. 
The reason behind lies in the flexibility of the numerical and that deduction of exact integrations scheme 
    to any given shape functions are not feasible. 
Due to the fact that the polynomials are adopted as the shape function, the numerical integrations methods 
    such as Legendre Quadrature or Gauss Quadrature provides possibility for an exact integration. 
An integration quadrature is normally defined as followed:
    \begin{equation}
        \int_{-1}^{1}
        f(x)dx 
        = \sum_{i=1}^n
        a_i f(x_i)
    \label{iso_eq:numerical_integration}
    \end{equation}

\paragraph{}
Any given targeted polynomial function defined on $[-1,1]$ can be explicitly expressed as series.
A set of integration points $\left\{ x_1, x_2, \dots, x_n \right\} \in \left[-1,1\right]$ and the corresponding weight $\left\{ a_1, a_2, 
    \dots, a_n \right\} \in \mathbb{R}$ determined from the integration quadrature can be adopted to perform an exact integration
    on the given function.
\paragraph{}
Although shape functions used in NURBS are not polynomials, they can be separated into several spans where the function is a 
    rational polynomial. 
Based on this property, we are able to apply the numerical integration quadrature on each of these spans and achieve a reasonably 
    accurate result.
In other words, the NURBS curve with a knot vector of 
$[ 
    \underbrace{-1,-1,\dots,-1}_{p+1}, 
    u_0,\dots,u_n, 
    \underbrace{1,1,\dots,1 }_{p+1}
]$
can be integrated as
\begin{equation}
    \int_{-1}^{1} R(u) du = \int_{-1}^{u_0} R(u)du + 
                            \int_{u_0}^{u_1} R(u)du + \dots +
                            \int_{u_n}^1 R(u)du
\label{iso_eq:numerical_integration_piecewise}
\end{equation}

\paragraph{}
Since the rational polynomials instead of the usual polynomials are utilized as the shape functions in NURBS, the difference between
    output from eq.~\ref{iso_eq:numerical_integration_piecewise} and the analytical solution will be so large that can not be regarded as
    machine error.
Based on eq.~\ref{iso_eq:rational_basis_function} we can conclude that the basis functions constructed by rational polynomials become
    non-rational if and only if the weight vector is identical i.e. $\left\{ w \right\} = \left[ 1,1,\dots,1 \right]$ after normalization.
That indicates the error of numerical integration will be decreased when the weight vector of the NURBS curves becomes more uniform as
    the basis functions are more close to non-rational polynomials.
In order to achieve this target, either or both of the knot insertion or the order elevation can be used.
\pagebreak