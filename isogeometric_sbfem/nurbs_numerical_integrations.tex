\section{Numerical integrations}
\label{iso_section:numerical_integration}
\paragraph{}
When computing the coefficient matrix (Eq.~\ref{lr_eq:sbfem_coe_matrix}) in SBFEM , numerical integrations tends to be overwhelmingly preferred over the mathematical deduction.
The reason behind lies in the flexibility of the numerical method and that deduction of exact integrations scheme to any given shape functions may not be feasible. 
Due to the fact that the polynomials are adopted as the shape function, the numerical integration methods such as Legendre Quadrature or Gauss Quadrature provides possibility for an exact integration numerically.
An integration quadrature is normally defined as followed:
    \begin{equation}
        \int_{-1}^{1}
        f(x)dx 
        = \sum_{i=1}^n
        a_i f(x_i)
    \label{iso_eq:numerical_integration}
    \end{equation}
%
The integration of any targeted polynomial function defined on $[-1,1]$ can be explicitly expressed as series.
A set of integration points $\left\{ x_1, x_2, \dots, x_n \right\} \in \left[-1,1\right]$ and the corresponding weight $\left\{ a_1, a_2, \dots, a_n \right\} \in \mathbb{R}$ determined from the integration quadrature can be adopted to perform an exact integration on the given function.
\paragraph{}
Although shape functions used in NURBS are not polynomials, they can be separated into several spans within which the basis functions are rational polynomials.
Based on this property, the numerical integration quadrature on each of these spans can be applied to achieve a reasonably accurate result.
In other words, the NURBS curve with a knot vector of 
$[ 
    \underbrace{-1,-1,\dots,-1}_{p+1}, 
    u_0,\dots,u_n, 
    \underbrace{1,1,\dots,1 }_{p+1}
]$
can be integrated as
\begin{equation}
    \int_{-1}^{1} R(u) du = \int_{-1}^{u_0} R(u)du + 
                            \int_{u_0}^{u_1} R(u)du + \dots +
                            \int_{u_n}^1 R(u)du
\label{iso_eq:numerical_integration_piecewise}
\end{equation}
%
Since the rational polynomials instead of the non-rational ones are utilized as the shape functions in the NURBS, the difference between the outputs from Eq.~\ref{iso_eq:numerical_integration_piecewise} and the analytical solution will be so large that can not be regarded as machine error.
Based on Eq.~\ref{iso_eq:rational_basis_function} we can conclude that the basis functions constructed by rational polynomials become non-rational if and only if the weight vector is identical i.e. $\left\{ w \right\} = \left[ 1,1,\dots,1 \right]$ after normalization.
That indicates the error of numerical integration will be decreased when the weight vector of the NURBS curve becomes more uniform as the basis functions are more close to the non-rational polynomials.
In order to achieve this target, the knot insertion or the order elevation introduced in Section~\ref{lr_sec:nurbs_knot_ins} and Section~\ref{lr_sec:nurbs_order_ele} can be used.
