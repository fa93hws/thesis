\subsection{Displacement interpolation}
\paragraph{}
Another difference between it with conventional FEM or SBFEM lies in the post processing.
After solving the partial differential equation numerically, the displacements on the nodes will be one of the output in
    the traditional method.
However, similar to what is discussed in \ref{subsection:surface_traction}, NURBS curves are defined by the control points
    that are not geometrically located on the curves.
As a consequence, not only the input such as surface traction need to be translated into a NURBS-like representation, the
    output such as the displacements will be the dummy values on the control points as well, or ``control displacements''
    $\left\{ u_c \right\}$.
Dislike that in the traditional method, the ``control displacements'' do not have any physical meaning. It can only be used
    to interpolate the real displacements within its span.

\begin{equation}
    \left\{ u \right\}=
    \sum_{i=0}^n
    R(u) \left\{u^{(N)}\right\}
\label{iso_eq:displacement_interpolation}
\end{equation}
\pagebreak