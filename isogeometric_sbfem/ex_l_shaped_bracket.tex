\subsection{L-shaped bracket}
\label{subsection:l_shaped_bracket}
\paragraph{}
In this example, consider an L-shaped bracket with isotropic material properties.
Fig. 14 shows the geometry and the boundary conditions of the problem.
The L-shaped bracket is fixed at one end and subjected to downward vertical displacement at the other end.
Plain strain conditions are assumed.
This problem was studied in \cite{LIPTON2010357} by employing the conventional IGA.
In their study, the fillet was modeled as a separate path using biquadratic NURBS with nine control points.
\paragraph{}
In the present study, the control mesh is directly employed for the stress analysis.
However, as the domain does not meet the star convexity, we divide the domain into three sub-domains 
    (see Fig. 14).
We employ NURBS to represent the fillet, whilst for the straight lines, we employ Lagrange basis functions.
The results from the present approach are compared with conventional finite element analysis using the commercial
    software ANSYS$^\circledR$.
\paragraph{}
A total of $2000$ $8$-node quadrilateral elements were used for the finite element analysis.
Fig. 15 shows the von Mises equivalent stress for the L-shaped bracket with and without the fillet.
As expected, the no fillet case shows higher stress when compared to the L-shaped bracket with the fillet.
From Fig. 15, it can be observed that the results from the present approach qualitatively match with the FE solution.
It should be noted that, the proposed method is computationally less intensive than the conventional IGA as it requires
    only the boundary information.



\pagebreak