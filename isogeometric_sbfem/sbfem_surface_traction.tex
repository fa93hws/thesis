\subsection{Surface traction}
\label{subsection:surface_traction}
\paragraph{}
In structural analysis, it is common to have boundary condition such as displacement constraints and applied load.
Due to the property of the NURBS that the control points are not necessarily on the curve, surface traction can not be
    applied by same method used in conventional numerical method like FEM or SBFEM.
A surface traction $\Phi$ can be regarded as Neumann boundary condition which can be expressed as
    \begin{equation}
        {F}=-\int_{\Gamma}
        [N]
        \Phi_n
        d\Gamma
    \label{iso_eq:neumann_bc}
    \end{equation}
where $[N]$ describe the shape functions and $\Phi_n$ is the surface traction on the nodes.

\paragraph{}
As mentioned in \ref{iso_subsection:numerical_integration}, numerical integration would be much more preferred over mathematical deduction
when the target function is an input. In the flavour of numerical integration, eq.~\ref{iso_eq:neumann_bc} can be expressed as followed.
    \begin{equation}
        {F}=-\sum_{i=1}^n
        a_i
        [N(\xi_i)]
        \Phi_n
    \label{iso_eq:neumann_bc_numerical}
    \end{equation}
Where $\xi_i$ is the integration points and $a$ is the weights,
$n$ is the number of integration points and different quadrature rule need different number to achieve a optimal accuracy.

\paragraph{}
It can be found that the term $[N(\xi_i)] \Phi_n$ is corresponding to $f(x)$ in eq.~\ref{iso_eq:numerical_integration}.
In conventional FEM or SBFEM, $\Phi_n$ can be determined as the real values on the nodes because geometrically speaking,
    its shape function is interpolated from the given set of points.
In other words, all nodes that determine the shape function in traditional FEM or SBFEM must be on the interpolating function.
However, this is not the case in NURBS curves where it is the control points that play the same role as the nodes in existing
    shape function.
    % figure required
In NURBS curves, apart from the first and the last points, the control points are not necessarily on the curves.
This prevent us from adopting the physical value on the nodes as $\Phi_n$ in eq.~\ref{iso_eq:neumann_bc_numerical}.
Instead, a set of ``control stress'' $\Phi_c$, the control points of another NURBS curve that represent the surface traction
    geometrically, need to be determined as \footnote{$\argmin_x f(x) = \left\{
        x | x \in S \wedge \forall y \in S : f(y) \geq f(x)
    \right\}$}
    \begin{equation}
        \Phi_c = \argmin_{\Phi_c}
            \frac{1}{2}
            \int_{-1}^1
            \|
                \Phi(\xi)-
                    \left[ N(\xi) \right]
                    \Phi_c
            \|^2
            d\xi            
    \label{iso_eq:surface_traction_fitting}
    \end{equation}

\paragraph{}
It means that ``control stress'' $\Phi_c$ describe a minimum mean squared error between surface traction NURBS curve and the real
    traction $\Phi$.
One of the simplest mathematical method to determine $\Phi_c$ will be least square method.
Given the fact that the shape functions of this NURBS curve will be the same as that describe the geometry, $\left[ N(\xi) \right]$
    can be considered as known.
By selecting $n$ sample points over the domain of the $\Phi$, eq.~\ref{iso_eq:surface_traction_fitting} can be rewrite as
    \begin{equation}
        \Phi_c = \argmin_{\Phi_c}
            \frac{1}{n}
            \sum_{i=1}^n
            \|
                \Phi(\xi_i)-
                    \left[ N(\xi_i) \right]
                    \Phi_c
            \|^2
    \label{iso_eq:surface_traction_fitting_discrete}
    \end{equation}
Then ``control stress'' $\Phi_c$ can be solved by least square as
    \begin{equation}
        \Phi_c= \left(
            \left[ N(\xi) \right] ^T
            \left[ N(\xi) \right]
        \right)^{-1}
        \left[ N(\xi) \right]^T
        \Phi(\xi)
    \end{equation}
and eq.~\ref{iso_eq:neumann_bc_numerical} in the case where NURBS is in use can be rewrite as
    \begin{equation}
        {F}=-\sum_{i=1}^n
        a_i
        [N(\xi_i)]
        \Phi_c
    \label{iso_eq:neumann_bc_numerical_NURBS}
    \end{equation}
\pagebreak