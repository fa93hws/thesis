\subsection{Convex hull in 2D}
\label{qdt_sc:convex_hull}
\paragraph{}
The convex hull property of the NURBS surface indicates that all points on the surface must be contained within the convex hull constructed by its control points \cite{SELIMOVIC2009772}
There are great number of algorithm that can be used including gift wrapping \cite{Cormen:2009:IAT:1614191}, graham scan \cite{ANDERSON197853}, quick hull \cite{Barber:1996:QAC:235815.235821}, Chan's algorithm \cite{Chan1996} and so on \cite{doi:10.1137/0215021, ANDREW1979216}.
The quick hull is adopted in the proposed as it provides a computationally efficient and stable algorithm.
The algorithm utilize the idea of `` divide and conquer'' to build the convex hull with an expected time complexity of $O(nlog(n))$ and  $O(n^2)$ for the worst case.
Generally speaking, it works as expected in most of the situation except for the case of high symmetry or most of the points located at the circumference of a circle.
The algorithm can be implemented with following steps:
\begin{enumerate}
    \item Find the most left and right points (points with minimal and maximum $x$) since they are proved to be part of the convex hull.
    \item Connect these two points and use the line to separate other points into two group.
    \item Find the point with maximum distance to the line in step 2 in any group.
    \item Construct a triangle with two points in step 2 and the point in step 3.
    \item Eliminate all points contained by these two subsets in step 4.
    \item Repeat the previous three steps and the distance calculated in step 2 is determined as the point to the triangle instead of the line in step 1.
    \item Terminate the iteration when no points are left
\end{enumerate}