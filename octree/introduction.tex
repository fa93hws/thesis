\section{Introduction}
\paragraph{}
The existing method is able to generate the mesh based on the STL file for a 3D model.
However, due to the fact that triangular surfaces in STL is not exactly on the original surfaces, most of the points on the boundary other than plane will be close to the origin surfaces but not exactly on it.
This geometric imperfection may have significant influence on the accuracy.
As a result, the algorithm that can extract geometric information directly from engineering design and the method that project points on the original boundary will be introduced.
Alternatively, the exact geometry can be retained by calculating the intersection between the line with the NURBS surface during the cutting stage of the mesh generation instead of perform a projection afterwards.

\paragraph{}
IGES introduced in Sec.~\ref{lr_sec:IGES} files will be extracted from the engineering design file in this chapter and information about NURBS surfaces and curves will be extracted from the files.
Points on the boundary from the mesh generated by the help of STL files will be projected back to the exact boundary surface by inversion algorithm in NURBS surfaces.
After that, SBFEM will be adopted to solve the problem.

\paragraph{}
This chapter will be organized as followed: some NURBS concept including efficient and robust algorithm that can project points back on to the exact surface followed by an introduction to calculate the intersection between a line and a NURBS surface.
After that, a brief introduction on SBFEM formulation in 3D elasticity is presented.
The accuracy and the convergence properties of the proposed method are demonstrated with benchmark problems in the context of linear elasticity.