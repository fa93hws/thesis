\section{Introduction}
\paragraph{}
This chapter starts with a mesh generated by the help of the octree based algorithm using STL file directly \citep{Liu2017}.
In this algorithm, the intersection is calculated between the edge of the element and the triangular surface which is an approximation of the exact geometry.
In order to achieve the geometric precision, a point projection method for 3D NURBS surface is presented.
It's computational efficiency is significantly improved by implementing a NURBS surface splitting and by utilizing the strong convex hull property of the NURBS.
The quick hull algorithm is also introduced to construct the convex hull from the control points in 3D.
Alternative method to retain the exact geometry is targeted as well.
The calculation of the intersection in \cite{Liu2017} is replaced by finding that between the edge of the element and the NURBS surface directly.

\paragraph{}
The advantage of the proposed method is that the exact geometry can be retained and hence improves the accuracy of the result.

\paragraph{}
This chapter will be organized as followed:
points projection of the NURBS surface is introduced at the beginning, together with the surface splitting and the convex hull construction.
After that, the calculation of the straight line with the NURBS surface is developed.
Furthermore, a brief introduction on SBFEM formulation in 3D elasticity is presented.
The accuracy and the convergence properties of the proposed method are demonstrated with benchmark problems in the context of linear elasticity.
Some other mesh examples from complex geometric input are also plotted at the end of this chapter.