\subsection{Convex hull in 3D}
\label{oct_sec:convex_hull}
\paragraph{}
The convex hull property of the NURBS surface indicates that all points on the surface must be contained within the convex hull constructed by its control points \citep{SELIMOVIC2009772}
The algorithm in 3D can be very similar to that in 2D as introduced in Sec.~\ref{qdt_sc:convex_hull}.
\begin{enumerate}
    \item Find the most left and right points (points with minimal and maximum $x$) since they are proved to be part of the convex hull.
    \item Connect these two points and use the line to separate other points into two group.
    \item Find the point with maximum distance to the line in step 2 in any group.
    \item Construct a triangle with two points in step 2 and the point in step 3.
    \item Find the point with maximum distance to the triangle in step 4. %5
    \item Construct a tetrahedron with the triangle in step 4 and the point in step 5.
    \item Follow the same procedures in Sec.~\ref{qdt_sc:convex_hull}
\end{enumerate}