\subsection{Matrix Representation for Rational Bezier Surface} 
\paragraph{}
A tensor-product rational Bézier surface of degree $(p_1,p_2)$ can be expressed as
\begin{equation*}
	\phi(u,v)\in\mathbf{R}^2 \rightarrow \frac{\sum_{i=0}^{p_1}\sum_{j=0}^{p_2}w_{i,j}\mathbf{P}_{i,j}B_i^{p_1}(t)B_j^{p_2}(t) }{\sum_{i=0}^{p_1}\sum_{j=0}^{p_2}w_{i,j}B_i^{p_1}(t)B_j^{p_2}(t)}
\end{equation*}
For the surface, $\mathbf{L}$ and $\mathbf{R}$ with order $(v_1,v_2)$
\begin{equation*}
	\mathbf{L} =
	\begin{bmatrix}
		B_0^{v_1+p_1}(u)B_0^{v_2+p_2}(v) & B_1^{v_1+p_1}(u)B_0^{v_2+p_2}(v) & \dots & B_{v_1+p_1}^{v_1+p_1}(u)B_{v_2+p_2}^{v_2+p_2}(v)
	\end{bmatrix}
\end{equation*}
\begin{equation*}
	\mathbf{R} =
	\begin{bmatrix}
		B_0^{v_1}(u)B_0^{v_2}(v)f_0(u,v) & B_0^{v_1}(u)B_1^{v_2}(v)f_0(u,v) & \dots & B_{v_1}^{v_1}(u)B_{v_2}^{v_2}(v)f_3(u,v)
	\end{bmatrix}
\end{equation*}
Following the same manner in the previous section, it can be derived that 
\begin{equation}
	\mathbf{S}_{\left( (i+k)(v_2+p_2+1)+j+l, l(v1+1)+k\right)} = 
	\frac{\mathbf{C}_k^{v_1}\mathbf{C}_l^{v_2}\mathbf{C}_i^{p_1}\mathbf{C}_j^{p_2}} 
		{\mathbf{C}_{i+k}^{v_1+d_2}\mathbf{C}_{j+l}^{v_2+d_2}}c_{(i,j)}
\end{equation}

\subsection{Property of $\mathbf{M_v}$ Matrix}
As described in the previous sections, the $\mathbf{M_v}$ matrix is defined so that
\begin{equation*}
	\begin{bmatrix}
	\psi_1(t_0) \dots \psi_{m_v}(t_0)
	\end{bmatrix}
	\times
	\mathbf{M_v(\mathbf{P})}
	= \vec{0}
\end{equation*}
where $\mathbf{P}$ is a point on the rational bezier curve/surface.
The order $v$ shall be no less than a critical value and it is proofed to be
\begin{itemize}
	\item $v>max(p-1,1)$ for rational bezier curve
	\item $(v_1,v_2) > (2p_1 -1, p_2 -1)$ or  $(v_1,v_2) > (p_1 -1, 2p_2 -1)$
\end{itemize}
The following properties are proofed in \cite{Laurent2014}
\begin{enumerate}
	\item For all degrees $geq$ critical degree and all point $\in\mathbf{R}^3$ , rank($\mathbf{M_v}(\mathbf{P})$) $<m_v$ if and only if $\mathbf{P}\in$ the closure of $\overline{Im}(\phi)$.
	\item If $\mathbf{P}\in\mathbf{R}^3$ is a point with a unique pre-image by $\phi$, the dimension of the null space of $\mathbf{M_v}(\mathbf{P})^T$ is one. 
	\item $\delta\mathbf{M_v}(\mathbf{P}) = 0$ if $\mathbf{P} \in\overline{Im}(\phi)$
\end{enumerate}
where 
\begin{equation*}
	\delta\mathbf{M_v}(\mathbf{P}) = \prod_{i=1}^{m_v}\sigma_i(\mathbf{M_v}(\mathbf{P}))
\end{equation*}
and $\sigma_i$ is the diagonal of $\Sigma$ in the SVD decomposition of $\mathbf{M_v}(\mathbf{P})=U\Sigma V^T$

\begin{enumerate}
	\setcounter{enumi}{3}
	\item $\forall\mathbf{P}\in\mathbf{R}^3$, $d(\mathbf{P},\overline{Im}(\phi))^{n_1}\leq c_1 \delta\mathbf{M_v}(\mathbf{P})$
	\item $\forall\mathbf{P}\in\mathbf{R}^3$, $\delta\mathbf{M_v}(\mathbf{P})^{n_2}\leq c_2 d(\mathbf{P},\overline{Im}(\phi))^{n_2} $
\end{enumerate}
where $c_1,c_2,n_1,n_2$ are constant.\\
These two properties give a distance function like function of the $M_v$ matrix. When the point get away to the surface, $\delta\mathbf{M_v}$ is getting larger and vice versa.vise visa.
\paragraph{}
If the point is on the curve/surface, in which case $\delta\mathbf{M_v}(\mathbf{P}) = 0$, the corresponding parameter value on the curve/surface can be easily found by a SVD numerically.\\
The computation of the null space of $\mathbf{M_v}(\mathbf{P}) $ will give a single vector $V=[v_1,v_2,\dots,v_{m_v}]$based on 2 and $V$ will be proportional to 
\begin{equation*}
	\begin{bmatrix}
	\psi_1(t_0) \dots \psi_{m_v}(t_0)
	\end{bmatrix}
\end{equation*}
More specifically, it will be proportional to
\begin{equation*}
	\begin{bmatrix}
		B_0^v & B_1^v & \dots
	\end{bmatrix}
\end{equation*}
for the rational bézier curves and
\begin{equation*}
	\begin{bmatrix}
		B_0^{v1}B_0^{v2} & B_0^{v1}B_1^{v2} & \dots
	\end{bmatrix}
\end{equation*}
for the rational bézier surfaces.