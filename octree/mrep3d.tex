\subsection{Matrix Representation for Rational Bézier Surface}
\label{oct_sec:mrep}
\paragraph{}
The intersection between a straight line and a NURBS surface is calculated using matrix representation method.
The method in 2D (NURBS curve) has been introduced in Sec.~\ref{adap_sec:mrep}.
A tensor-product rational bézier surface of degree $(p_1,p_2)$ can be expressed as
\begin{equation}
	\phi(u,v)\in\mathbf{R}^2 \rightarrow \frac{\sum_{i=0}^{p_1}\sum_{j=0}^{p_2}w_{i,j}\mathbf{P}_{i,j}B_i^{p_1}(t)B_j^{p_2}(t) }{\sum_{i=0}^{p_1}\sum_{j=0}^{p_2}w_{i,j}B_i^{p_1}(t)B_j^{p_2}(t)}
\end{equation}
%
For the surface, $\mathbf{L}$ and $\mathbf{R}$ with order $(v_1,v_2)$
\begin{subequations}
\begin{align}
	\mathbf{L} =
	\begin{bmatrix}
		B_0^{v_1+p_1}(u)B_0^{v_2+p_2}(v) & B_1^{v_1+p_1}(u)B_0^{v_2+p_2}(v) & \dots & B_{v_1+p_1}^{v_1+p_1}(u)B_{v_2+p_2}^{v_2+p_2}(v)
	\end{bmatrix} \\
	\mathbf{R} =
	\begin{bmatrix}
		B_0^{v_1}(u)B_0^{v_2}(v)f_0(u,v) & B_0^{v_1}(u)B_1^{v_2}(v)f_0(u,v) & \dots & B_{v_1}^{v_1}(u)B_{v_2}^{v_2}(v)f_3(u,v)
	\end{bmatrix}
\end{align}
\end{subequations}
%
Following the same manner in Sec.~\ref{adap_sec:mrep}, it can be derived that 
\begin{equation}
	\mathbf{S}_{\left( (i+k)(v_2+p_2+1)+j+l, l(v1+1)+k\right)} = 
	\frac{\mathbf{C}_k^{v_1}\mathbf{C}_l^{v_2}\mathbf{C}_i^{p_1}\mathbf{C}_j^{p_2}} 
		{\mathbf{C}_{i+k}^{v_1+d_2}\mathbf{C}_{j+l}^{v_2+d_2}}c_{(i,j)}
\end{equation}
%
\subsection{Properties of the $\mathbf{M_v}$ Matrix}
As described in the previous sections, the $\mathbf{M_v}$ matrix is defined so that
\begin{equation}
	\begin{bmatrix}
	\psi_1(t_0) \dots \psi_{m_v}(t_0)
	\end{bmatrix}
	\times
	\mathbf{M_v(\mathbf{P})}
	= \vec{0}
\end{equation}
where $\mathbf{P}$ is a point on the rational bézier curve/surface.
The order $v$ shall be no less than a critical value and it is proofed to be
\begin{itemize}
	\item $v>max(p-1,1)$ for rational bézier curve
	\item $(v_1,v_2) > (2p_1 -1, p_2 -1)$ or  $(v_1,v_2) > (p_1 -1, 2p_2 -1)$
\end{itemize}
The following properties are proofed in \citep{Laurent2014}
\begin{enumerate}
	\item For all degrees $geq$ critical degree and all point $\in\mathbf{R}^3$ , rank($\mathbf{M_v}(\mathbf{P})$) $<m_v$ if and only if $\mathbf{P}\in$ the closure of $\overline{Im}(\phi)$.
	\item If $\mathbf{P}\in\mathbf{R}^3$ is a point with a unique pre-image by $\phi$, the dimension of the null space of $\mathbf{M_v}(\mathbf{P})^T$ is one. 
	\item $\delta\mathbf{M_v}(\mathbf{P}) = 0$ if $\mathbf{P} \in\overline{Im}(\phi)$
\end{enumerate}
where 
\begin{equation}
	\delta\mathbf{M_v}(\mathbf{P}) = \prod_{i=1}^{m_v}\sigma_i(\mathbf{M_v}(\mathbf{P}))
\end{equation}
and $\sigma_i$ is the diagonal of $\Sigma$ in the SVD decomposition of $\mathbf{M_v}(\mathbf{P})=U\Sigma V^T$

\begin{enumerate}
	\setcounter{enumi}{3}
	\item $\forall\mathbf{P}\in\mathbf{R}^3$, $d(\mathbf{P},\overline{Im}(\phi))^{n_1}\leq c_1 \delta\mathbf{M_v}(\mathbf{P})$
	\item $\forall\mathbf{P}\in\mathbf{R}^3$, $\delta\mathbf{M_v}(\mathbf{P})^{n_2}\leq c_2 d(\mathbf{P},\overline{Im}(\phi))^{n_2} $
\end{enumerate}
where $c_1,c_2,n_1,n_2$ are constant.\\
These two properties give a distance function like function of the $M_v$ matrix. When the point get away to the surface, $\delta\mathbf{M_v}$ is getting larger and vice versa.
\paragraph{}
If the point is on the curve/surface, in which case $\delta\mathbf{M_v}(\mathbf{P}) = 0$, the corresponding \hl{parametric} value on the curve/surface can be easily found by a SVD numerically.\\
The computation of the null space of $\mathbf{M_v}(\mathbf{P}) $ will give a single vector $V=[v_1,v_2,\dots,v_{m_v}]$based on 2 and $V$ will be proportional to 
\begin{equation}
	\begin{bmatrix}
	\psi_1(t_0) \dots \psi_{m_v}(t_0)
	\end{bmatrix}
\end{equation}
More specifically, it will be proportional to
\begin{equation}
	\begin{bmatrix}
		B_0^v & B_1^v & \dots
	\end{bmatrix}
\end{equation}
for the rational bézier curves and
\begin{equation}
	\begin{bmatrix}
		B_0^{v1}B_0^{v2} & B_0^{v1}B_1^{v2} & \dots
	\end{bmatrix}
\end{equation}
for the rational bézier surfaces.
%
The calculation of the intersection is described in detail in \citep{Buse2010} and \citep{Ba2009}.
All intersections can be calculated at once by using matrix representation of the algebraic curve/surface.
\paragraph{} 
Given a rational curve/surface $C1$
\begin{equation}
	\mathbf{P}^1 \xrightarrow{\phi_1} \mathbf{P}^n: (u,v) \rightarrow(f_0,f_1,f_2,f_3)(u,v)
\end{equation}
the aim is to find the intersection via matrix representation method between it with another ration curve $C2$
\begin{equation}
	\mathbf{P}^1 \xrightarrow{\phi_2} \mathbf{P}^n: (t) \rightarrow(g_0,g_1,g_2,g_3)(t)
\end{equation}
is to find
\begin{equation}
	\mathbf{M}_{v1}(\phi_2(t) = 0
\end{equation}
which leads to
\begin{equation}
	\mathbf{M_0}g_0 + \mathbf{M_1}g_1 + \mathbf{M_2}g_2 + \mathbf{M_3}g_3 = 0
	\label{mRep_intec_base}
\end{equation}
By knowing $g_n$ is a polynomial function with order $p$, Eq.~\eqref{mRep_intec_base} can be rearranged as
\begin{equation}
	\mathbf{M}(t) = \sum_{i=0}^p \mathbf{M_i}t^i
\end{equation}
After that, the generalized companion $q\times p$-matrices $A,B$ with rank $\rho$are introduced
\begin{equation}
	A = 
	\begin{bmatrix}
		0		&I 		&\dots 		&\dots 		&0 		\\
		0 		&0 		&I 			&\dots 		&0 		\\
		\vdots 	&\vdots &\vdots 	&\vdots 	&\vdots \\
		0 		&0 		&\dots		&\dots 		&I 		\\
		M_0^t 	&M_1^t 	&\dots 		&\dots 		&M_{d-1}^t
	\end{bmatrix}
\end{equation}
\begin{equation}
	B = 
	\begin{bmatrix}
		I 		&0 		&\dots 		&\dots 		&0 		\\
		0 		&I 		&0 			&\dots 		&0 		\\
		\vdots 	&\vdots &\vdots 	&\vdots 	&\vdots \\
		0 		&0 		&\dots 		&I 			&0 		\\
		0 		&0 		&\dots 		&\dots 		&-M_d^t \\		
	\end{bmatrix}
\end{equation}
Before the eigenvalues are calculated, the regular part of a non-square pencil of the matrices shall be extracted first which is done by the following step
\paragraph{} 
\textbf{Step 1}
	Transform B into its column echelon form by SVD-decomposition.
	\begin{equation}
	\begin{aligned}
		B_1 = BV_0 = [\underbrace{B_{1,1}}_{\rho} |\underbrace{0}_{q-\rho}]	\\
		A_1 = AV_0 = [\underbrace{A_{1,1}}_{\rho} |\underbrace{A_{1,2}}_{q-\rho}]
	\end{aligned}
	\end{equation}
\paragraph{} 
\textbf{Step 2}
	Transform $A_{1,2}$ into its row echelon form
	\begin{equation}
		U_1A_{1,2} = 
		\begin{bmatrix}
			\underline{A^\prime_{1,2}}\\
			0
		\end{bmatrix}
	\end{equation}
	where $A^\prime_{1,2}$ is in full row rank.\\
	At the end of step 2, matrix $A$ and $B$ can be represented as
	\begin{equation}
	\begin{aligned}
	A^\prime_1 &=
	\begin{bmatrix}
		A^\prime_{1,1} & A^\prime_{1,2} \\
		\cmidrule(lr){1-2}
		A_2 & 0
	\end{bmatrix}\\
	B^\prime_1 &=
	\begin{bmatrix}
		B^\prime_{1,1} & 0\\
		\cmidrule(lr){1-2}
		B_2 & 0
	\end{bmatrix}
	\end{aligned}
	\end{equation}
where $A^\prime_{1,2}$ has full row rank\\
$
\begin{bmatrix}
\underline{B^\prime_{1,1}}\\
B_2
\end{bmatrix}
$ has full column rank\\
$
\begin{bmatrix}
\underline{B^\prime_{1,1}}\\
B_2
\end{bmatrix}
$ and $B_2$ are in echelon form
\paragraph{}
$A_2$ and $B_2$ will be the new $A$ and $B$ matrices for next iteration until $B$ has full rank. If $B$ has full row rank but not full rank, $A=A^T$, $B=B^T$.
\paragraph{}
After these process, $A$ and $B$ become two square matrices and $B$ is invertible so that the solution for the intersection parameter $t$ can be determined from the eigenvalue of the matrix $AB^{-1}$
