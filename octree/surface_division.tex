\subsection{Surfaces division}
\label{oct_sc:surface_division}
\paragraph{}
Mapping points back to NURBS surfaces in 3D can be extremely time consuming as there is no known close form mathematical solution.
Every point takes about ten to hundreds iterations before the nearest projection point on the NURBS surface is found, depending on the size of the projection surface.
However, in the problem that the proposed method is targeting, reasonably complex geometry is expected.
As a result, points projection back to such kind of NURBS surfaces may takes much more computational time than any others do and it may be necessary to find a more complicated but computational efficient algorithm other than the naive implementation.

\paragraph{}
One concept that can be utilized to improve the efficiency here is the ``divided and conquer''.
As the time complexity of the naive algorithm is $O(n^3)$ where $n$ is directly correlated to the order of the basis function and the number of control points used to describe the NURBS surfaces, dividing a surface into two generally will make the projection algorithm four times faster.
Consequently, breaking the origin NURBS surfaces into multiple smaller ones could be one of the practical practices.

\paragraph{}
Surfaces division can be performed by the help of knot insertion (Sec.~\ref{lr_sec:nurbs_knot_ins}).
Assuming a NURBS surface defined by two knot vectors\\
$
\Xi_1 = [-1, -1, -1, a_1, a_2, \dots, a_n , 1, 1, 1]
$\\
and
$
\Xi_2 = [-1, -1, -1, b_1, b_2, \dots, b_m, 1, 1, 1]
$.\\
Several knots will be inserted into these two vector so that all interior knots will repeated $p+1$ times and $p$ stands for the order of the NURBS basis function in that direction.
After knot insertion, the same NURBS surface will now be described by two new vectors\\
$
\Xi_1^\prime = [-1, -1, -1, a_1, a_1, a_1, a_2, a_2, a_2, \dots, a_n , 1, 1, 1]
$\\
and
$
\Xi_2^\prime = [-1, -1, -1, b_1, b_1, b_1, b_2, b_2, b_2, \dots, b_m, 1, 1, 1]
$.\\
Extraction then can be conducted by taking the sub-matrix from the generated control points matrix $P^\prime$ and weight matrix $w^\prime$.

\paragraph{}
Fig.~\ref{oct_fig:nurbs_division} shows a sub-division of breaking a cylinder surface into four smaller ones.

\begin{figure}[h!]
    \centering
    \begin{subfigure}[b]{0.4\linewidth}
        \centering
        \scalebox{0.27}{
            \includegraphics{octree/images/NURBSParent.png}
        }
        \caption{Original NURBS surface}
    \end{subfigure}
    \begin{subfigure}[b]{0.4\linewidth}
        \centering
        \scalebox{0.25}{
            \includegraphics{octree/images/NURBSChildren.png}
        }
        \caption{Subdivided child NURBS surfaces}
    \end{subfigure}
    \caption{NURBS surface subdivision}
    \label{oct_fig:nurbs_division}
\end{figure}
