\paragraph{}
The FEM could be one of the most popular numerical method\hl{s} in engineering.
A necessary procedure is to discretize the problem domain into FE mesh of elementary shapes.
In 3D problem, the conventional FEM allows hexahedron, tetrahedron, wedge and pyramid only which limits the flexibility of mesh generation.
In order to achieve a reasonably accurate result, the mesh of the traditional FEM is required to conform to the boundary of the problem domain.
As a consequence, it could be necessary to develop an automatic mesh generation algorithm using limited types of shapes \citep{Frey:2007:MGA:1205626}.
The development of an automatic generation algorithm is reported to be able to save 80\% of the overall analysis time \citep{Hug2005}.

\paragraph{}
The research toward automatic mesh generation is popular over decades \citep{owen2000,Blacker1993,doi:10.1002/fld.1650081003,doi:10.1093/comjnl/24.2.167}.
Generally speaking, the hexahedral element is favored over the tetrahedral element in terms of accuracy but it could be difficult to generate the mesh with hexahedral elements only automatically.
Methods using plastering \citep{Blacker1993}, whisker weaving \citep{Tau1984}, sweeping \citep{Staten1999} and octree \citep{doi:10.1002/nme.1620201103} were proposed to generate a hexahedral mesh.
However, meshing of arbitrary domains using hexahedral element\hl{s} without losing the exact geometric representation is not achievable up to now.
Methods based on tetrahedral elements using Delaunay triangulation \citep{doi:10.1093/comjnl/24.2.167} and the advancing front technique (AFT) \citep{doi:10.1002/fld.1650081003} were proposed for arbitrary domains.
But Delaunay triangulation requires boundary recovery \citep{doi:10.1002/nme.808,LIU201432} and AFT need to solve colliding fronts \citep{Shewchuk1997}.
Furthermore, a mix use of all allowed elements \citep{owen2000} was proposed while the accuracy is not as satisfactory as that determined from the all-hexahedron mesh.

\paragraph{}
As a consequence, new numerical methods that reduce the limitation on element usage including X-FEM \citep{Moes1999}, isogeometric analysis \citep{Hug2005} (introduced in Sec.~\ref{lr_sec:iso_analysis}), finite cell method (FCM) \citep{Parvizian2007} and SBFEM \citep{Wol2003} (introduced in Sec.~\ref{lr_sec:sbfem}) were proposed.

\paragraph{}
The X-FEM eases the burden posed on mesh generation as conforming to the geometric boundary is not necessary.
Geometric discontinuity within the element is solved by the adoption of partition of unity \citep{MELENK1996289} and enrichment functions.
The method is then extended by the help of level-set \citep{OSHER198812} to be able to solve the holes \citep{Sukumar2001}, problems with material interfaces \citep{doi:10.1002/nme.2259} and flows \citep{Chessa2003}.
It has been applied in field of fluid-structure-contact interaction problems \citep{Mayer2010} and crack propagation for elastostatic problems \citep{doi:10.1002/nme.429,doi:10.1002/nme.430} in 3D.

\paragraph{}
In FCM, the mesh is generated by simple unfitted structured mesh of higher-order basis functions and the geometry is represented by averaging the adaptive quadrature points which removes the necessity of boundary conforming.
The method has been improved in terms of topology \citep{Parvizian2012} and applied to voxel model \citep{doi:10.1002/nme.3289} and problems with material interfaces \citep{Joulaian2013}.

\paragraph{}
Recently, an automatic mesh generation algorithm developed from the SBFEM and the octree-based algorithm using STL file (introduced in Sec.~\ref{lr_sec:stl}) was proposed \citep{Liu2017}.
Arbitrary element faces are allowed in the scaled boundary finite element which significantly decreases the limitation of element shape.
Mesh generated from an octree-based algorithm leads to higher quality elements and higher computational efficiency.
However, the geometry can not be retained exactly in the STL file which means the inevitable geometric imperfection may result in considerable accuracy issue \citep{Hug2005}.

