
\paragraph{}
In the proposed approach, the geometry and the unknown fields are represented by non-uniform rational B-splines (NURBS).
In this section, we give a brief overview of NURBS.
For more detailed description and implementation aspects, interested readers can refer to \cite{Pie1997,NGUYEN201589}.
\paragraph{}
% basis function
NURBS are the superset of B-spline functions.
B-spline is short for basis spline and is a generalization of Bézier curves.
A spline function is a piecewise polynomial function of degree $p$ and the points of intersection of such functions are called knots.
The number of knots must be equal to or greater than $p + 1$.
One of the salient features of spline functions is that the functions are continuous at the knots, however, the continuity of the functions can be altered by repeating the knots.
The B-spline functions are parametric functions of the form $F(\eta)$ in which the parameter $\eta$ lies in the parametric space.
The key ingredients in the construction of B-spline functions are: the knot vector (a non decreasing sequence of parameter values, $\eta_i \leq \eta_{i+1}$ , $i = 0,1,\dots,m -1$) and the degree of the curve p. The $i$th B-spline basis function of degree $p$, denoted by $_{i,p}$ is defined as \cite{Pie1997}:
\begin{equation}
    \begin{aligned}
        N_{i,0}(\eta) &=
            \begin{cases}
                1   & \text{if } \eta_i \leq \eta \leq \eta_{i+1}    \\
                0   & \text{else}
            \end{cases}\\
        N_{i,p}(\eta) &= 
            \frac{\eta - \eta_i}{\eta_{i+p}-\eta_i}     N_{i,p-1}(\eta) - 
            \frac{\eta_{i+p+1}-\eta}{\eta_{i+p+1} - \eta_{i+1}}     N_{i+1,p-1}(\eta)
    \end{aligned}
    \label{lr_nurbs_basis}
\end{equation}

The first derivative of the B-spline basis function can be computed recursively from lower order basis functions as:
\begin{equation}
    \frac{d}{d\eta} N_{i,p}(\eta) =
        \frac{p}{\eta_{i+p} - \eta_i} N_{i,p-1}(\eta) -
        \frac{p}{\eta_{i+p+1} - \eta_{i+1}} N_{i+1,p-1}(\eta)
\end{equation}

\paragraph{}
% basis properties
The B-spline basis functions has the following properties:
\begin{enumerate}
    \item Non-negativity
    \item Partition of unity, $\sum_i N_{i,p}=1$
    \item Interpolatory at the end points. The last point requires special treatment when imposing non-homogeneous Dirichlet boundary conditions [58].
\end{enumerate}


\paragraph{}
% curves
Moreover, the spline function has limited support.
Given $n + 1$ control points $(P_0 ,P_1,\dots,P_n )$ and a knot vector 
    $\Xi$ = $\left\{
        \eta_0 ,\eta_1 ,\dots,\eta_m 
    \right\}$, the piecewise polynomial B-spline curve of degree p is defined as:
\begin{equation}
    C(\eta) = \sum_{i=0}^n P_i N_{i,p} (\eta)
\end{equation}

where $P_i$ are the control points.
A B-spline curve has the following information: $n+1$ control points, $m+1$ knots and a degree $p$.
It is noted that $n$,$m$ and $p$ must satisfy $m = n + p + 1$.
The B-spline functions also provide a variety of refinement algorithms, which are essential when employing B-spline functions to discretize the unknown fields.
The analogous $h$ and $p$ refinement can be done by the process of `knot insertion' and `order elevation'.
Another unique feature of the B-spline basis function is that, it is possible to combine the knot insertion and the degree elevation, commonly referred to as ‘k-refinement’ in the literature \cite{Hug2005b}.
Here we briefly discuss the knot insertion and the degree elevation.
For more details, interested readers are referred to \cite{Pie1997,Hug2005b} and references therein.


\subsubsection{Knot Insertion}
\paragraph{}
Consider a B-spline basis functions defined on $\Xi = \left\{
    \eta_0 ,\eta_1,\dots,\eta_m 
    \right\}$, let $\overline{\eta} \in [\eta_k ,\eta_{k+1} )$, and insert $\overline{\eta}$ into $\Xi$ to form a new knot vector $\overline{\Xi} = \left\{
    \eta_0 ,\dots, \overline{\eta}_k = \eta_k , 
    \overline{\eta}_{k+1} = \overline{\eta}, 
    \overline{\eta}_{k+2} = \overline{\eta}_{k+1} ,
    \dots,\eta_{m+1} = \eta_m
    \right\}$. Simultaneously, the size of the control points is increased by one. Thus $C(\eta)$ has a representation on $\overline{\Xi}$ of the form