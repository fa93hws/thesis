\paragraph{}
In a traditional approach, geometric design and analysis were treated as separate modules requiring different methods and interpretations.
For example, the geometric design module employed non-uniform rational B-splines (NURBS) introduced in Sec.~\ref{lr_sec:NURBS} to describe the geometry, whilst the analysis module consisted of one of the following
\begin{enumerate}
    \item Mesh based discrete models, such as the finite element method (FEM) \citep{doi:10.1111/j.1467-8667.1989.tb00025.x}
    \item Boundary based methods, such as the boundary element method (BEM) \citep{book}, scaled boundary finite element method (SBFEM) \citep{Son1997}
    \item Meshless methods \citep{article123213eds}
\end{enumerate}
The approximation space employed in the analysis module to describe both the geometry and the fields is different from that used in the CAD system.
Hence it requires repetitive conversion between the CAD and the analysis and in this process errors are inevitable.
Moreover, the analysis module employs polynomials that do not lead to exact representation of the geometry, whilst the geometric module employs Bézier representations that use Bernstein polynomials or B-splines and NURBS that employ de Boor polynomials \citep{Pie1997}.
The above representations utilize basis functions and control points to represent the geometry, in addition to this, B-splines and NURBS also utilize a vector
of knots.
NURBS further use weights to control points to model intricate shapes.

\paragraph{}
As a consequence, the concept of isogeometric analysis is proposed \citep{Hug2005}, in which the conventional Lagrange polynomials are replaced with the NURBS basis functions.
The concept of isogeometric analysis (IGA) has revolutionized the analysis procedure.
The IGA provides a natural link with the CAD model.
A key feature of this framework is that the geometry is represented exactly by NURBS and the isoparametric concept is invoked to define the field variables.
Since its inception, the method has been applied to a variety of problems such as plates and shells \citep{NGUYENTHANH20113410,NGUYENXUAN2014222,HOSSEINI20141}, as
cohesive elements \citep{NGUYEN2014193}, for shape optimization \citep{WALL20082976}, fluid–structure interaction problems \citep{BAZILEVS201228}, problems with strong discontinuities and singularities \citep{doi:10.1093/imamat/hxu004, doi:10.1002/nme.4580, BAZILEVS201228}, optimization problems \citep{GHASEMI2014463} to name a few.
Jia et al. \citep{JIA2013342} by incorporating reproducing kernel approximation methods, alleviated the instabilities of the conventional triangular B-spline element.
The new approach yielded improved convergence rate and accuracy when compared to the conventional triangular B-spline element.
This seems to be a promising alternative to NURBS and T-splines where considerable effort is required for local refinements. 
In the conventional IGA, the surfaces/volumes are represented by the tensor product of the corresponding knot vectors.
This requires the domain to be discretized with standard shapes and leads to a restricted number of boundary curves/surfaces.
Also, this leads to excessive overhead of control points with refinement.
This can be circumvented by adopting local refinement as proposed \citep{NGUYENTHANH20111892} or by employing T-splines \citep{Sederberg:2003:TT:882262.882295}.
Recently, Simpson et al. \citep{Sim2013, SIMPSON201287} proposed the isogeometric boundary element method (IGABEM), in which the NURBS functions were used to approximate the unknown fields.
This framework circumvents the need to discretize the domain, as required by the IGAFEM.
It was shown that the IGABEM is more accurate than the conventional BEM with polynomial interpolations.
Furthermore, Scott et al. \citep{Sco2013} and Simpson et al. \citep{SIMPSON2014265} combined the collocated IGABEM with T-splines for linear elastostatics and acoustic analysis, respectively.
The concept of IGABEM was further extended to damage tolerance assessment \citep{PengXuan;AtroshchenkoElena;Bordas2014} and shape sensitivity analysis \citep{LianHaojie;SimpsonRobert;Bordas2013}.