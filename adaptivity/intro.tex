\paragraph{}
The main objective of this chapter is to introduce a way that can improve the overall mesh quality by refinement automatically and adaptively.
Unnecessary refinement in the region which contributes little to the improvement to the accuracy will be largely prevented.
The expressions related to the eigen-values of the SBFEM formulation representing the quantity of the error in the interpolation are adopted as one of the error indicators, together with the area and other geometric properties of the Scaled Boundary Finite Element.
A machine learning model using the Multilayer Perceptron (MLP) is trained to determine whether a Scaled Boundary Finite Element needs refinement or not based on all these information.

\paragraph{}
The proposed method enhances the SBFEM with quad-tree mesh introduced in Sec.~\ref{qdt_sec:main} and the outstanding features of the method are:
\begin{itemize}
    \item No human effort involvement in mesh refinement
    \item Smart refinement detection
    \item Flexible refinement criteria
\end{itemize}

\paragraph{}
This chapter is organized as follows.
The error indicators used in the proposed method will be introduced first.
After that, a machine learning algorithm that can be trained to determine the necessity of the refinement of a cell based on these indicators.
Furthermore, a triangle merging algorithm is developed to bypass the lack of some error indicators in first order triangle element.\
Finally, the matrix representation of NURBS curves is introduced to improve the computational efficiency and the robustness of the NURBS related calculation.
The accuracy and the convergence properties of the proposed techniques are demonstrated with benchmark problems in the context of linear elasticity, followed by concluding remarks in the last section.
